\documentclass[12pt,a4paper]{article}
\usepackage{verbatim}
\usepackage{graphicx}
\usepackage{amsmath}
\usepackage{float}
\author{ZHANG Xiao Research Intern in IBM CRL}
\title{Network 20q HW4}

\begin{document}
\maketitle
\pagebreak

\section{Voting}
\subsection{Plurality voting}
In plurality voting, only the first place of the positional voting counts, so in this voting 8/31 voters put A first, 12/31 put B first and 11 voters put C first. So the voting result is $B>C>A$.
\subsection{Condorcet voting}
While adapting this method, we need to count every validate pairwise comparison count, so for we could have the following counts:
\begin{equation}
\begin{tabular}{ l c r }
\hline
Comparison Preference & Count \\
\hline
 $A>B$ & 9+8+0 = 17\\
 $B>A$ & 7+5+2 = 14\\
 $A>C$ & 8+5+0 = 13\\
 $C>A$ & 9+7+2 = 18\\
 $B>C$ & 8+7+5 = 20\\
 $C>B$ & 9+2+0 = 11\\
\hline
\end{tabular}
\end{equation}
So we could get $A>B$,$C>A$ and $B>C$, then we could get a cycle as $A>B>C>A$.
\subsection{Borda Count}
In this method, every first place in the voting could get 2 points, second place could get 1 points, and the final result will be extracted from the points each candidates got.

Then $P_A= 9*1+8*2+7*0+5*1+2*0+0*2 = 30$, $P_B=9*0+8*1+7*2+5*2+2*1+0*0=34$, $P_C=9*2+8*0+7*1+5*0+2*2+0*1=29$. Because $P_B>P_A>P_C$, we could conclude that $B>A>C$.

\section{Perturbing flipping behaviors}

a. In the first graph, while we start with p(0) = 0.01, the line would perturb up to the next equilibrium, that is $\frac{1}{3}$ which meet with the 45-degree line. So $p_\infty = \frac{1}{3}$.

b. Under this condition, the equilibrium at 0 do not exist any more, but the line would still perturb up to the next equilibrium, that is $\frac{1}{3}$ which meet with the 45-degree line. So $p_\infty = \frac{1}{3}$

c. In this graph, the equilibrium at $\frac{1}{3} and \frac{2}{3}$ do not exist any more, so when flip up, the line would not encounter the 45-degree line until when $p=1$. So $p_\infty = 1$.

\end{document}